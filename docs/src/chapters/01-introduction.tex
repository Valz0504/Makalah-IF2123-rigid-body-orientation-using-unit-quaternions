\section{Introduction}

The orientation of objects in three-dimensional space is a fundamental topic in linear algebra and geometry, with important applications in computer graphics, robotics, aerospace engineering, and computer vision. Accurately modeling the orientation of a rigid body is essential for describing rotational behavior while preserving geometric properties such as distances and angles. A mathematically consistent framework is therefore required to ensure that rotations remain rigid and well-defined in three-dimensional space.

Traditional approaches to modeling orientation often rely on Euler angles, which describe rotation as a sequence of axis-aligned rotations. Although intuitive, Euler angles suffer from inherent limitations, most notably gimbal lock, a singularity that results in the loss of one rotational degree of freedom. This limitation can cause ambiguity and instability when describing rigid-body orientation, particularly in scenarios involving continuous or compounded rotations.

Unit quaternions offer a mathematically robust approach for describing three-dimensional orientation. By extending complex numbers into four dimensions and enforcing a unit-norm constraint, unit quaternions represent rotations as normalized algebraic entities. This formulation avoids singularities, preserves orthogonality, and allows rotations to be composed efficiently through quaternion multiplication. Consequently, unit quaternions have become a widely adopted tool for modeling rigid-body orientation in both theoretical and computational contexts.

This paper focuses on rigid-body orientation using unit quaternions from linear algebra and geometric perspective. The discussion is restricted to rotational motion in three-dimensional Euclidean space, without considering translational motion or dynamic control. By examining the mathematical foundations of unit quaternions and their role in describing rigid-body orientation, this paper aims to provide a clear and accessible understanding of their effectiveness in modeling three-dimensional rotations.